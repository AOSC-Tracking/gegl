The GIMP project has many developers, all with different interests and located
around the world. To work within this setting, new contributors should be
willing to participate in the normal development process of the project, and
demonstrate their interest and committed to the advancement of the project.  

This starts with communication via emails, GIMP news groups, developer lists,
documentation, proposals, and IRC as well. GIMP interest groups at conferences
and SIGGRAPH are other places where developers and users can meet.  

For the most part there is no direct management in the traditional sense for
open source projects. No managers tell anyone they must do something.  Rather
people and interested parties offer to contribute. This means it is up to
potential contributors to learn about what is going on by looking at
documentation (user and programmer) and studying code and reading emails. 

Once developers prove they can contribute (usually by coding things like
bug fixes as patches, or writing plug-ins) they graduate to working on libraries
and similar things by finding areas that need work or maintenance, and by
volunteering to do that work.

Just as GIMP development must address the needs of the software itself it must
also improve the development process within the GIMP project. This involves the
need for concrete design documents for each module of GIMP, and schedules and
task lists that describe ongoing status of the library or module. An example of
this can be seen for the GEGL library, and hopefully as other parts of GIMP 2.0
modules are built, they will include similar documentation as well.

Please feel free to contact the authors with any questions or issues and ideas
about the above and becoming involved with GIMP and GEGL.

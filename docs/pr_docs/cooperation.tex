The GIMP project has a number of main core developers, all interested in different areas and located in different places. To work within this setting, new contributors should be willing to participate in the normal development process of the project, and demonstate that they are interested and committed to the advancement of the project.
  
This starts with communication via emails, GIMP news groups, developer lists, documentation, proposals, and IRC as well. GIMP interest groups at conferences and SIGGRAPH are other places where developers and users meet regularly. 
 
For the most part there is no direct management in the traditional sense though. No managers will tell anyone they must do something. Rather people and interested parties offer to contribute. This means it is up to potential contributors to learn about what is going on by looking at documentation (user and programmer) and studying code and reading emails. 

Once developers prove they can contribute (usually by coding things like bugfixes as patches, or writing plugins) they graduate to working on libraries and similar things by finding areas that need work or maintanance and proposing and taking the initiative to take care of those areas.

Just like it is time to bring more structure to GIMP's internal, it is necessary to introduce more structure to the development cycle of GIMP. This means that there needs to exixts design documents for each componment of GIMP, and a schedule of all the tasks that needs to be done to complete the library. We have already started this process with GEGL and all the other libraries will follow shortly. 

 
((Based on the schedule in the previous chapter there are plenty of areas that need contributions. There are lots of opportunity both in the main areas of GIMP and GEGL for helping out. Experienced graphics programmers interested in contributing should have no trouble finding areas that need contributors and making progress.)) 


((However to start with, it might be of great help to have serious contributors actually meet some core GIMP programmers and discuss many of the development issues and investigate how and where they can fit into the project. Any of the authors of this document could help investigate that possibility. This might help in understanding how GIMP can fit into a particular production environment as well. ))

Please feel free to contact the authors with any questions or issues and ideas about the above and becoming involved with GIMP and GEGL.

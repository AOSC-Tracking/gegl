To make sure that all our efforts satisfies all our needs in the shorts amount of time it is essential that we have a good understanding of what the specifications are and how to going about to realize them. 

We have presented a tentative plan of what is needed to get a version of GIMP ready to be tested by users. Currently our cost estimate of this is approximated 4000 man-hours. The more people that join GIMP team the faster we will see this new version provided that we have a organized way of diving up the work amongst ourselves.

Because all of us will be working from different places and probably will not have the ability to go down the hall to talk to the other member to clarify a point it is essential that we have a clear vision of how all the parts work together before we all start coding.

We propose that we all (current gimp team and new team mates) meet in person to clarify all the points. During this meeting we will finalize the specification, the schedule and dived up the project between us.

In order to make the meeting a success it is important to come prepared. We would want all the involved member have a clear vision of what they need and have a understanding of the current design. 

All communication between us after we have left the meeting will happen through an email and through documentation that we post to the web. We want to have weekly status reports of all the members to make sure that we are not faced with any surprises further down the line. Maybe bi-monthly meeting would be a good way to reevaluate and make sure we are on the right track.

We like to point out that we have no intention of converting GIMP into a tool for just the film industry. GIMP should continue to be a leader amoung the 8 bit paint programs but with the added feature of being able to handle various data types and color models, thereby making it a tool that can be used in film production.



 

It is essintial that this merger of GIMP with the film industry does not result in the film industry taking over the GIMP. We want the film industry to join the GIMP team and not to take over it.
 
 





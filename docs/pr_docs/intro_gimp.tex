
Gimp 1.2, due to be formally released ``Real Soon Now'' is a direct
descendent of of the 1.0.4 of 1998-1999. A Gimp developer napping from
the time of the 1.0 release would, upon awakening, find essentially
the same design patterns that characterized 1.0 -- just more code
arising from it.

It would be code filled with good intentions. In the core, she would
discover that a certain amount of separation has occurred between the
pipeline and the rest of the application. Much pipeline activity now
occurs in the context of GTK+ idleloop, where a list of damage
rectangles, rather than direct tool coupling, update the display; the
user interface is now more responsive. She would find the tools far
more ``objectified'' and abstracted, with standard methods that the
event loop call that resolve into different implementations, giving
tools their distinctive character. She would find some new tools as
well, A clone tool, measure tool, a smudge tool, and she would find
some old tools, such as Bezier paths, much improved. She would find
a comprehensive support library furnishing a good framework for plugin
developers. She would also find the user interface much improved, with a more
consistent look and feel and a multitude of thoughtful touches and
able to speak fourteen or fifteen languages with the switch of an
environment variable. She would find hundreds of incremental improvements
on the same basic framework

But she would find lots and lots of UI code intimately entwined among
the four hundred and forty seven header and implementation files
constituting ``the core proper,'' with no clear separation between the
two. And, if, in the course of her nap, she had suffered from amnesia,
she would find that two or three months of study would be needed
before she could offer anything in the line of diagnostic patches or
bug fixes. One of the authors of this paper is a relative newcomer to
the team, and had he not been on sabbatical and able to invest four or
five hours a day over a similar period, he would not be writing these
lines now. The price of entry into a support role would have been too
much in terms of time required.

 
It has been said that ``The road to hell is paved with good intentions.''
and this is certainly the case with the 1.2 Gimp code base. The original
framework furnished by Spencer Kimball and Peter Mattis is still in
place, but now carrying a good deal more baggage -- all of it lovingly
supplied by people with the best of intentions. There is very little 
separation in the design, a great deal of state lies in global variables,
so it has become very difficult for a newcomer -- or even an individual
with a few year's experience on the code base -- to make changes 
without triggering possibly extensive collateral effects.

This has transformed GIMP into a singularly unresponsive application.
The pre-release ``feature freeze'' is now over a year old, discouraging
those who would like to invest time into feature development.
 Bug fixing must proceed with care, and it is not uncommon for a bug fix 
to introduce collateral bugs. No one dares contemplate merging various
GIMP CVS branches into a unified application. 

Karl Fogel has observed in ``Open Source Development with CVS''
(Designing for Decentralized Development) that good design in the Open
Source sense may not actually make programs run more efficiently, or
improve the quality or accuracy of their output, but it should make
the general plan of the program easy to grasp.  It was clear by the
end of 1999 that what GIMP needed was not so much a new tool or
plugin, but a reorganization in the way the code read as well as the
way the code functioned; that code comprehensibility is a necessary
aspect of social engineering. This observation has lent force to
the modularization of GIMP, a central concept at the first GimpCon,
held in Berlin.


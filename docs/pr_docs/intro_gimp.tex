
Though the upcoming 1.2 release of GIMP contains many new features, most
development effort since the 1.0 version has involved improved ui and tools,
and little improvement in underlying architecture. Internally GIMP is still a
monolithic application with a plug-in library wrapped around it. It was never
broken into libraries or manageable pieces. 

This makes it difficult to add things like color management, CMYK support,
different color models and data types. Without breaking GIMP into a series of
libraries, moving forward with any major architectural enhancements is
impossible. So the plan for GIMP after 1.2 release is to break GIMP into
modules, separate the UI from the engine code, and give GIMP the flexibility to
handle the next stage of functionality. Unfortunately there is no way to avoid
this with GIMP's current state.

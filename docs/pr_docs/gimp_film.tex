The film version of GIMP supports both 16-bit and floating point channel data.
This version was written as a test for introducing high color resolution
support into future versions of GIMP, and to investigate the viability of using
GIMP for film production work. Work on this version was done as a branch of the
main GIMP project. This branch is called HOLLYWOOD (the cvs tag for it).
HOLLYWOOD is based on the GIMP 1.0.4 codebase, and the necessary changes were
made to that version to allow it to work for 16bit and float channels. The
HOLLYWOOD branch was updated to work with a more recent version of the GIMP ui
toolkit (GTK+1.2) as well.  More information about the film version of GIMP is
available at http://film.gimp.org.

This film version of GIMP is being used currently at Rhythm and Hues Studios.
(see Chapter 4).

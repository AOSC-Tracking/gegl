In June 2000 the first official GIMP Developers Conference took place
in Berlin. This conference was sponsored by the Free Software
Foundation, the Chaos Computer Club, and O'Reilly Germany and
organized by Sven Neumann and Michael Natterer. At the meeting most of
the core developers met to discuss the future direction of GIMP.

Out of three days of roundtable discussion, the twenty participants
concluded that the current architecture of GIMP made it difficult to
move forward with the plans for color management, CMYK support, and
high color resolution channels. The GIMP core code was not broken into
independent modules or libraries after reaching 1.0 and had become
difficult to work with. 

The general plan coming out of that conference was to make GIMP more
general and more modular allowing new functionality to be added with
greater ease. The particulars of how is will be done is still a matter
of open discussion among core GIMP developers, and some concepts, such
as GEGL are more clearly defined than others. The remainder of this
chapter discusses the general plan. GEGL, which is more clearly
defined than the remaining components, is discussed in chapter
\ref{ch:GEGL}, ``'GEGL.'


The schedule for achieving GIMP 2.0 is broken into three main parts. 

The first part involves the development of GEGL. Since GEGL is a low-level
library to be used by GIMP, work for it can proceed somewhat independent of
GIMP.    

Once GEGL is far enough along, a simple skeleton of imaging and painting tools
will be developed using GEGL. This stripped down version is called MicroGIMP
and will include some simple paint, retouch and editing tools, enough to
provide first tests of each of the major areas of functionality for GIMP:
painting, editing, compositing, color conversions. 

After MicroGIMP is built and tested as a small production tool, the rest of the
major GIMP libraries will be built. These libraries include the various GIMP UI
libs and the core code for GIMP's graphics objects (Images, Layers, Channels,
Drawables). GIMPs task management (rendering pipeline), libraries for
plugin communication  as well as GIMPs component for scripting (PDB,
Procedural Database) are all included here.  

\begin{flushleft}		
\begin{tabular}{|p{2cm}|p{5cm}|p{2cm}|l|}\hline
\multicolumn{4}{|c|}{\rule[-3mm]{0mm}{8mm} \large \bf Schedule for GIMP 2.0}\\  
\hline
Task & Description & Library & Cost\\ 
\hline 

\multicolumn{4}{|c|}{\bf MicroGIMP}\\
\hline
MicroGimp-skeleton & Create stripped down GIMP with basic simple toolbox ui. & gimpcore & 40h\\
\hline
MicroGimp-basic classes & Write initial simple versions of Image, Layer, Drawable classes for stripped down GIMP. & gimpcore & 40h\\
\hline
MicroGimp-paint core & Write the initial paint core code for microgimp using GEGL. & gimpcore & 80h\\
\hline
MicroGimp-tools & Write first test tools: paint, clone, simple edit tools  & gimpcore & 120h\\
\hline

\multicolumn{4}{|c|}{\bf GEGL}\\
\hline
\raggedright{Area\\ Operation} & Classes for common area operations in GEGL (convolve, kernel ops) & gegl & 120h\\
\hline
\raggedright{Geometric\\ Operations} & Classes for common geometric operations such as scaling, transformation. & gegl & 120h\\
\hline
\raggedright{GIL\\ Specialization} & Mechanism for allowing partial specialization in GIL code & gegl & 120h\\
\hline
\raggedright{Point\\ Operations} & Classes for LUT operations & gegl & 120h\\
\hline
\raggedright{Preprocessor for\\ codegen} & Generates colormodel and data type code from GIL specifications & gegl & 80h\\
\hline
\raggedright{Memory \\Management} & \raggedright{Cache and virtual memory management classes} & gegl & 300h\\
\hline
\raggedright{Multi-thread \\support} & \raggedright{Multi-threaded support for image operations.} & gegl & 200h\\  
\hline

\multicolumn{4}{|c|}{\bf GIMP}\\
\hline 
\raggedright{Image, Drawable,\\ Layer, Channel} & Complete Image, Drawable, Layer, Channel classes. The basic core classes used in GIMP & gimpcore & 160h\\
\hline 
Widgets UI & Build utility widgets for use in other UI libs  & gimpwidgets & 200h\\
\hline
Core UI & Private widgets not shared with plugins & gimpcoreui & 120h\\
\hline
Gimp UI & Widgets shared with plugins & gimpui & 200h\\
\hline 
Painting tools & Paint, Airbrush, Erase, Blur etc & gimpcore, gimpui & 120h \\
\hline 
\raggedright{Editing \\tools} & Rect Select, Free select, Bezier etc & gimpcore, gimpui & 120h \\
\hline 
\end{tabular}

\begin{tabular}{|p{2cm}|p{5cm}|p{2cm}|l|}\hline
\multicolumn{4}{|c|}{\rule[-3mm]{0mm}{8mm} \large \bf Schedule for GIMP 2.0}\\  
\hline
Task & Description & Library & Cost\\ 
\hline 
\multicolumn{4}{|c|}{\bf GIMP(cont)}\\
\hline 
Resize tools & Scale, Resize, Transform... & gimpcore, gimpui & 80h \\
\hline 
\raggedright{Image\\ Adjustment tools} & Brightness, Levels, Histogram & gimpcore, gimpui & 120h\\
\hline
\raggedright{GIMP\\ Render Pipeline} & GIMP task management code to manage calls to gegl, ui tasks, io & gimpcore, gegl & 120h \\
\hline 
Generic UI & Widgets for multiple data types & gimpwidgets, gimpcoreui, gimpui & 120h\\
\hline
Undo & Build undo history mechanism & gimpcore & 100h\\
\hline
\raggedright{Plugin\\libraries} & Corba based design for communication between plugins and GIMP & gimpext, gimpiface & 300h\\
\hline

\multicolumn{3}{|r|}{Total man-hours} & 3100\\
\hline
\end{tabular}
\end{flushleft}

	

Since its founding, Rhythm \& Hues has had a history of writing proprietry
software tools for production needs. These tools were written because they
didn't exist at the time, or available tools could not be altered or customized
easily, or because of license restrictions and costs of buying software. Rhythm
\& Hues has proprietary rendering, animation, modeling, and compositing
packages, all supported by a software staff. In addition various proprietary
paint packages have been used in the past as well.

Having access to the source code for production software has had some obvious
advantages for Rhythm \& Hues, among them being able to alter tools for exact
needs of production tasks, and also insure that the tools work together in a
larger pipeline reliably, efficiently and without data loss throughout.

This is why GIMP is a perfect choice for Rhythm \& Hues. Since it is open
source it can be treated effectively as an in-house software package, supported
by the software department, customized for image and data types and installed
on every machine as well, just like proprietary packages.

Rhythm \& Hues also uses a custom image and data format, and mixing
off-the-shelf packages always involves data conversion and possible precision
loss. With GIMP, R\&H can eliminate the color conversion loss problem because
it is easy to customize the internals of the software. Because of this it is
possible to edit and retouch just one individual frame of a scene, without
worrying about clipping or data loss, or running a script that simple opens and
saves every frame in a scene because you have to edit one frame.

Other big advantages of using GIMP include allowing technical directors at
Rhythm and Hues to install and use multiple display film look up tables for
viewing film images on monitors. TDs can view images with a variety of these
tables depending on the type of job, or film stock.

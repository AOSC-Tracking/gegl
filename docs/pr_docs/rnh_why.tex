Rhythm \& Hues has a history of writing proprietary software tools for
production needs. These were written because commercial versions didn't exist at
the time, or available tools could not be altered for production needs or were
costly to install on every machine. Rhythm \& Hues has in-house rendering,
animation, modeling, and compositing packages, all supported by a software
staff. In addition, various proprietary and commercial paint packages have been
used in the past as well.

Having access to the source code for production software has some obvious
advantages, among them being able to alter tools for exact needs of production
tasks, and also insure that tools work together in a larger pipeline without
data loss.

This is why GIMP is a perfect choice for Rhythm \& Hues. Because it is open
source, it can be installed, compiled and supported by in-house programmers.
This is a great advantage over commercial paint solutions, which rarely can be
configured to match the Rhythm \& Hues data types, and so involve some kind of
color data loss to use, as well as being costly to install on every machine.

Other big advantages of having source code for GIMP include being able to
install and use multiple film display look up tables for viewing film images on
monitors. TDs can view images with a variety of these tables depending on the
type of job, or film stock.

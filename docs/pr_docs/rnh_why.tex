Rhythm \& Hues has a history of writing proprietary software tools for
production needs. These were written because commercial versions didn't exist
at the time, or available tools could not be altered for production needs.
Also it is expensive to install commercial versions of software on large
numbers of machines.

Having access to the source code for software has some obvious advantages as
well, among them the ability to alter tools for exact needs of production
tasks, and to insure that tools work together in a pipeline without data loss. 

This is why GIMP was a good choice for Rhythm \& Hues. It can be installed,
compiled and supported by in-house programmers in much the same way proprietary
software can.  This is in contrast to commercial paint solutions, which rarely
can be configured to match the Rhythm \& Hues image data types.

Other big advantages of having source code for GIMP include being able to
install and use multiple film display look up tables for viewing film images on
monitors. TDs can view images with a variety of these tables depending on the
type of job, or film stock.

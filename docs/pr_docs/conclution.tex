Open Source or Off the Shelf? We think that GIMP
2.0 is the best alternative for the long
run. Here's how our thinking goes.

\section{On Being Beholden To Other Peoples' Engineering}

With a commercial paint package, one has binaries
only.  There are few open-ended variables; a
per-seat price can be established with relative
ease, training and documentation is available from
the vendor. If a good many seats are purchased, a
studio may even have leverage over future design.

But, by in large, commercial packages do not sell
intellectual property; they sell right to use a
piece of software engineering built according to
someone else's assumptions, and these are not
uniform across the industry, giving rise to image
mis-match problems.

For example, If a single frame from a scene must
be edited in a package whose format doesn't match
the native format of the image file, to ensure
consistent color throughout, every frame of the
entire scene must be opened, then saved, in that
paint package. The cause of such tedium arises
from differences in white point and dynamic ranges
embedded in the various 16-bit/channel data types
common in film production. With each studio using,
or each package built around its own version of
these data types, finding a paint package that can
handle a particular type or choice and that can be
installed on the various flavors of UNIX can be a
daunting task.

\section{Abstract Image Data Types and Modular Design}

Properly built around GEGL, GIMP become thoroughly
unaware of differences among image data types;
GEGL abstracts pixel, image, and drawable data
types, so the actual bit manipulation occurs at
the lowest level; the choice of the particular 16
bit formats that GIMP consumes is very localized.
Any individual aware of the prevailing studio
formats can modify GIMP to suit that environment.
This is just one example of the benefit accruing from
not being captive to third party engineering.

\section{Is There Life In GIMP Senior?}

Unfortunately, the current GIMP 1.x does not offer
similiar flexibility. Adapting GIMP 1.0.4 to 16
bits was not a ``local change.'' The task of
merging this variant back into the 1.1.x branch is
daunting and forcefully brings home the importance
of modular design and motivates the core
developers to rewrite the application along
modular lines.

The ``cost'' of this is not as easy to estimate 
as that of a shrinkwrapped package, and the intent
to reengineer suggests years of effort instead of
months. This may lead one to the possibility of 
``fixing'' the current GIMP, perhaps merging 
HOLLYWOOD deep color with the 1.1.x main GIMP branch

Our hopes for this are sanquine. The difficulty of
adapting the code (Chapter \ref{ch:GIMP_2.0}) is,
in itself, an effort that could take many months,
and would lack the flexibility of a GEGL-based
GIMP, since in the 1.0.x and 1.1.x GIMP series,
information about image data types are not
well-contained; they are, in fact, quite
global. It may be possible to merge the HOLLYWOOD
1.0.x and 1.1.x GIMP series before a 2.0
deployment, but the combined product will remain a
a monolithic application with a small plug-in
library wrapped around it, and demanding months of
just learning curve time from individuals tasked to
support the code base.

\section{A Call for Participation}

Quite frankly, we hold greater hope for a complete
rewrite, and we have come to trust the
multiplicative effects of Open Source developers
who have become engaged in The Game again. With
this proposal -- a call for participation, actually --
we hope to draw in people outside the current core, 
people who bring with them fresh insights and fresh
points of view. We feel that a design well-adapted to
distributed development, drawing in people from an
industry that faces the hardest image challenges, 
can greatly reduce (for want of a better term ``time
to market.''

We hope that you will join us in contributing to
the development of GIMP. GIMP is a great example
of an open source program whose advancement and
progress is clearly beneficial for the graphics
community at large. From a very practical point of
view, it can be made to solve many of the problems
that turn up with paint programs in production
settings. It is a solution that can last as well.

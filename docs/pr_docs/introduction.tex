This paper discusses GIMP (General Image Manipulation Program) and its possible
use in film and production environments. GIMP is a freely available open source
paint program (http://www.gimp.org). It can be downloaded with source code, and
built and installed on most Unix and Windows machines. It has a standard set of
painting and image editing tools, similar to Adobe Photoshop. It supports
plug-ins and scripting.

Because the source code is available, GIMP can be customized to fit the needs of
production. Experimental versions which support 16-bits per channel have been
built and proven useful in production for over a year (see Appendix B).

With the unstable future of many of the current 16-bit off-the-shelf paint
packages on the Unix platform, GIMP has a definite advantage for the film
industry. It provides an opportunity to build and contribute to a stable and
customizable solution for many of the particular needs of production. 

This document describes GIMP and gives some details about the upcoming
architectural changes for future versions. It also discusses ways interested
parties can help contribute to the GIMP community.  

The following is an outline of the topics we cover:

\begin{itemize}

\item Chapter \ref{ch:GIMP_2.0_Design} ``GIMP 2.0 Design'' describes 
the basic architecture of GIMP 2.0.

\item Chapter \ref{ch:GEGL} ``GEGL'' reviews the most concretely realized 
component of GIMP 2.0, the GEGL library. GEGL will be the heart of
the GIMP 2.0 image processing engine.

\item Chapter \ref{ch:Schedule} ``Schedule'' is a list of tasks for GIMP 2.0
which describes the main parts of the next major version of gimp. It gives some
rough schedule estimates for the tasks.  

\item Chapter \ref{ch:Conclusion}, ``Conclusion'' summarizes our document. 

\end{itemize}

In addition, the appendices cover some useful extra material:

\begin{itemize}

\item Appendix  \ref{ch:GIMP_Versions_Licences} ``GIMP Versions, Licenses''
covers GIMP versions 0.54 to 1.2 and the experimental HOLLYWOOD branch. It also
describes GIMP's license.
  
\item Appendix  \ref{ch:GIMP_at_Rhythm_Hues}  ``GIMP at Rhythm \& Hues
Studios'' goes into details of the use of the GIMP HOLLYWOOD branch in a
production setting. 

\item Appendix \ref{ch:Contributing_to_GIMP} ``Contributing to GIMP'' This is a
short description of how things get done in open source projects.

\item Appendix \ref{ch:More_Information} ``More Information'' contains URLs to
web pages with more information.

\end{itemize}

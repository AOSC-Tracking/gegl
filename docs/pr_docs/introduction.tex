The General Image Manipulation Program (GIMP) ~(http://www.gimp.org) is an
open source paint program that is freely available, can be adapted and used in
many different situations and can be installed on most Unix and Windows
machines. The eight bit version is a staple of the Linux desk top; the Windows
community is finding GIMP an inexpensive alternative to consumer paint programs.

With the unstable future of many of the current 16-bit off-the-shelf paint
programs, GIMP is also an attractive option for the film industry - adapted,
of course, to the special needs of that industry. 

\begin{enumerate}

\item because it is open source paint program, GIMP is free in the sense of
intellectual freedom. Permission does not need to be sought to modify
the program to fulfill particular purposes. Recently, GIMP and Rhythm
\& Hues developers ported a version of GIMP 1.0.4 to use floating
point/16-bit data types. This 16-bit version of GIMP was intended to
test its capabilities in a film production environment and to
investigate changes needed to support high bit depths in future
versions of GIMP. In conducting these tests, Rhythm \& Hues developers
were not captive to the development shedule of third-party
vendors. They understood their needs and commenced exploratory
development.

Although still missing some important features, this version has
proven that GIMP has the potential of becoming a serious tool for film
and high color resolution work.  This film version is currently used
in production at Rhythm \& Hues Studios. Appendix \ref{ch:GIMP_at_Rhythm_Hues} 
``GIMP at Rhythm \& Hues Studios'' describes GIMP's
current and projected use there.

\item GIMP is among one of the most actively maintained open source projects
on the planet. There are active mailing lists and chat channels
dedicated to GIMP users and developers. There are a dozen GIMP book
titles that support the rank beginner and the graphics professional,
and more titles are due. This constitutes a virtual support
organization which makes no annual support contract demands. Competent
support from a core GIMP developer or power user is typically one
e-mail away.

\end{enumerate}

But GIMP is not free in a monetary sense. 

Rhythm \& Hues invested programmer talent, floor space, and equipment,
all costs to its business. Further, at some juncture it deployed its
experimental GIMP on a project; that program had to add value before
deadline arrival, so, to some degree, R \& H put a production schedule
at risk, and the back-out and fallback planning also constituted a
production cost. And, more generally, every individual involved in the
support of GIMP contribute time and talent. This expenditure is not
without cost to these individuals. Presumably these individuals get
something back for their effort. For many professionals, recognition
of the ``neat hack'' accrues an award of recognition that money can't
buy. 

Obviously, costs distribute themselves differently in an Open Source
project, challenging bean counting and project management customs, but
costs clearly exist, and there is no reason to believe they cannot
become substantial. As with any effort, costs associated with
deploying GIMP have to be identified and weighed against benefits.

We believe the benefits accrued in using GIMP far outweigh the costs,
and this document outlines why we believe this to be so. The only
question in our minds is whether the benefit arrives sooner or
later. To that end, we discuss the expected costs of future GIMPs and
suggest how an infusion of help from the film industry can bring those
benefits to the table sooner rather than later.

\begin{enumerate}

\item Chapter \ref{ch:GIMP_2.0} ``GIMP 2.0 Design Concepts'' reviews the 1.2 
release and discusses why it is not the ``out-of-box'' GIMP for either
the film industry or future GIMP development.  Reviews the GimpCon
2000 conference, which arrived at the basic architecture of GIMP
2,0. Not only can this GIMP be the platform for the general consumer,
but it can be the platform for professional industries as well.

\item Chapter \ref{ch:GEGL} ``GEGL'' reviews the most concretely realized 
component of GIMP 2.0, the GEGL library. GEGL will be the heart of
the GIMP 2.0 rendering pipeline 

\item Chapter \ref{ch:Schedule} ``Schedule'' reviews what we think GIMP 2.0
will cost in terms of man-hours. We may be missing scaling factors
here, but we think we have the relative proportions correct.

\item Chapter \ref{ch:Conclusion}, ``Conclusion'' claims that 
GIMP 2.0 will be written. Being anxious kids at heart, we'd like it
written sooner rather than later, and we think principle GIMP users
feel the same way too. We even think that principle GIMP users
can accelerate GIMP 2.0's gestation period.

\end{enumerate}

For background, we offer a number of appendices:

\begin{enumerate}

\item Appendix  \ref{ch:GIMP_Versions_Licences} ``History of Gimp'' is a thumbnail of GIMP
from version 0.54 to 2.0, the version first discussed at GimpCon 2000
in Berlin. A concluding note on the GNU General Public License, which
control's Gimp's distribution

\item Appendix  \ref{ch:GIMP_at_Rhythm_Hues}  ``GIMP at Rhythm \& Hues Studios''
goes into greater detail of the use of HOLLYWOOD GIMP in a production
setting. GIMP did not meet all needs of all people, it was found unable to
slice bread, but it did remove wires and more. That experiment gave the 
authors the courage to write this document.

\item Appendix \ref{ch:Contributing_to_Gimp} ``Contributing to Gimp''
considers how studios could contribute resources (defined as
programmers, time, tools, documentation, or users) to help further the
GIMP project and what kind of work is involved with this effort.

\item Appendix \ref{ch:More_Information} ``More Information'' is an
edited version of our favorite links page ;) \end{enumerate}


For high color resolution film work the ability to edit frames in a paint
package with minimal color data loss is extremely important. If a single frame
from a scene must be edited in a package whose format doesn't match the native
format of the image file, it means the entire scene must be opened and saved in
that paint package to ensure consistent color throughout.  Problems like these
are common in working with 16bit/channel data types in film production. Each
16bit data type makes a different choice for dynamic range, white point and so
on. With each studio or package with its own version of these data types
finding a paint package that can handle a particular type or choice and can be
installed on every machine is difficult.

The General Image Manipulation Program (GIMP) ~\cite{http://www.gimp.org} is an
open source paint program that is freely available, can be adapted and used in
many different situations and can be installed on most Unix and Windows
machines. This makes it a perfect candidate to solve problems like the above.

Recently, GIMP and Rhythm \& Hues developers ported a version of GIMP 1.0.4 to
use floating point/16-bit data types. This 16-bit version of GIMP was intended
to test GIMP in a film production environment and to investigate changes
needed to support high bit depths in future versions of GIMP. Although still
missing some important features, this version has proven that GIMP has the
potential of becoming a serious tool for film and high color resolution work.
This film version is currently used in production at Rhythm and Hues Studios.

With the unstable future of many of the current 16-bit off-the-shelf paint
programs, GIMP is an attractive option because it is open source, and actively
maintained. This document outlines how studios could contribute resources
(defined as programmers, time, tools, documentation, or users) to help further
the GIMP project and what kind of work is involved with this effort.

Section 2 describes current GIMP versions and plans for future versions.
Section 3 describes the image processing library GEGL, intended for future
versions of GIMP. Section 4 describes how GIMP is used in production now at
Rhythm and Hues Studios. Section 5 describes how studios could contribute
resources and help further the development of GIMP as a serious production
paint tool.

This document is an invitation to join an already existing group of GIMP
contributers, programmers and supporters. While we are describing some uses of
GIMP for film production use, as with most open source projects there are
interests of all kinds involved. It is clear though that a general purpose
actively maintained and configurable paint tool is in the interest of everyone.

The General Image Manipulation Program (GIMP) ~(http://www.gimp.org)
is an open source paint program that is freely available, can be
adapted and used in many different situations and can be installed on
most Unix and Windows machines. The eight bit version is a staple of
the Linux desk top; the Windows community is finding GIMP an
inexpensive alternative to consumer paint programs.

With the unstable future of many of the current 16-bit off-the-shelf paint
programs, GIMP is also an attractive option for the film industry -- adapted,
of course, to the special needs of that industry. 

\begin{enumerate}

\item because it is an open source paint program, GIMP is free in the sense of
intellectual freedom. Permission does not need to be sought to modify
the program to fulfill particular purposes. Recently, GIMP and Rhythm
\& Hues developers ported a version of GIMP 1.0.4 to use floating
point/16-bit data types. In conducting these tests, Rhythm \& Hues developers
were not captive to the development schedule of third-party
vendors. They understood their needs and commenced exploratory
development. This film version is currently used
in production at Rhythm \& Hues. Appendix \ref{ch:GIMP_at_Rhythm_Hues} 
``GIMP at Rhythm \& Hues Studios'' describes the program's
current and projected use.

\item GIMP is among one of the most actively maintained open source projects
on the planet. There are ongoing mailing lists and chat channels
dedicated to GIMP users and developers. There are a dozen GIMP book
titles that support the rank beginner and the graphics professional,
and more titles are due. This constitutes a virtual support
organization which makes no annual support contract demands. Competent
support from a core GIMP developer or power user is typically one
e-mail away.

\end{enumerate}

But GIMP is not free in a monetary sense. 

Rhythm \& Hues invested programmer talent, floor space, and equipment,
all costs to its business. Further, at some juncture it deployed its
experimental GIMP on a project; that program had to add value before
deadline arrival, so, to some degree, R \& H put a production schedule
at risk.

More generally, every individual involved in the support of GIMP
contribute time and talent. This expenditure is not without cost to
these individuals. Peculiar to the Open Source community, should these
individuals determine that their personal costs are outweighing a
(largely psychological) return, they lose interest in the project,
slowing the pace of development and putting the user base at risk.
Combating that risk is important, but we think it is beatable and
we'll address it at the end of this introduction.

Obviously, costs distribute themselves differently in an Open Source
project, challenging current bean counting and project management
customs, but costs clearly exist, and there is no reason to believe
they cannot become substantial. As with any effort, costs associated
with Open Source projects have to be identified as containable, their
magnitudes have to be estimated and weighed against benefits.

This document advocates that GIMP can deliver benefits that exceeds
costs. First, the benefits can be optimized to meet particular aims,
for in the Open Source community, aims are specified by those who
participate most vigorously in the project and who demonstrate
competence by hitting at what they aim. There are no marketing
departments subjugating feature sets to product demographics. Second,
the costs are identifiable. These include traditional time costs to
develop code (estimated in Chapter \ref{ch:Schedule}), and the
less-traditional costs of integrating with an Open Source community
(Discussed in Appendix \ref{ch:Contributing_to_GIMP}, ``Contributing
to GIMP.'')

The greatest risk facing any Open Source project -- the risk
that determines if anyone benefits at all -- is the risk of malaise.
Participants in Open Source projects are not primarily motivated by
money. Many are already richly compensated (thank you); they 
are in it for something else; they are in it for ``The Game.''
The Game is partly competitive -- to be the first to implement
the neat hack -- and it is partially cooperative, for the 
camaraderie of teaming with colleagues to hit targets deemed
unreachable is not quite like any elixir that money can buy, and
once imbibed is addictive. The sense that a project is 
done, that there is no room anymore for neat hacks -- that The
Game is Over -- dissipates Open Source teams faster than any
other effect.

This risk, while grave, is beatable. As this document describes,
GIMP 2.0 has room for a lot of neat hacks. Its emphasis on modular
design makes not only technical but social sense, because
the pieces can be ``owned'' by particular people. Contrast
this to the current GIMP 1.2 code base, where changes in certain 
areas are unpredictably global, bringing the perpetrator angry
e-mail and disapproval, often discouraging further participation. 

To the extent that GIMP will benefit anyone sooner or later depends
very much on furnishing a point of nucleation just as the good people
are hovering around, wondering ``What's next?'' GIMP 1.2 is done; GIMP
2.0 is in the wings. Our goal with this document is to attract the
people who work every day with the really hard image problems -- the
people in the film industry -- and who know how to spell their aims
out. We don't need very many; just a few around whom the many will
gravitate. The best thing that can come of this is that the really
hard problems are also the most attractive, and those who will benefit
first will be those who frame the hunt. There are risks involved, many
beyond the somewhat murky crystal ball of this document, but the very
worse one somehow goes away when people are just too damn busy to
notice.

So, are you game?  

If so, then here are our best guesses as to the problems,
and what needs to be done.

\begin{enumerate}

\item Chapter \ref{ch:GIMP_2.0} ``GIMP 2.0 Design Concepts'' reviews the 1.2 
release and discusses why it is not the ``out-of-box'' GIMP for either
the film industry or future GIMP development.  Reviews the basic architecture 
of GIMP 2,0.

\item Chapter \ref{ch:GEGL} ``GEGL'' reviews the most concretely realized 
component of GIMP 2.0, the GEGL library. GEGL will be the heart of
the GIMP 2.0 rendering pipeline.

\item Chapter \ref{ch:Schedule} ``Schedule'' reviews what we think GIMP 2.0
will cost in terms of man-hours. We may be missing scaling factors
here, but we think we have the relative proportions correct.

\item Chapter \ref{ch:Conclusion}, ``Conclusion'' is where we 
summarize what we think we've said.

\end{enumerate}

For background, we offer a number of appendices:

\begin{enumerate}

\item Appendix  \ref{ch:GIMP_Versions_Licences} ``History of GIMP'' from
0.54 to 2.0, Also notes on the GNU General Public License, which
control's GIMP's distribution.

\item Appendix  \ref{ch:GIMP_at_Rhythm_Hues}  ``GIMP at Rhythm \& Hues Studios''
goes into greater detail of the use of HOLLYWOOD GIMP in a production
setting. GIMP did not meet all needs of all people, it was found unable to
slice bread, but it did remove wires and more. It's the experiment that gave the 
authors the courage to write this document.

\item Appendix \ref{ch:Contributing_to_GIMP} ``Contributing to GIMP''
Anyone who has managed a software development team in a traditional sense
can find Open Source strange. The compensation is rich, but it isn't always
money. Notes on how things get done in Open Source projects.

\item Appendix \ref{ch:More_Information} ``More Information'' is an
edited version of our favorite links page ;) \end{enumerate}
